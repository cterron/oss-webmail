\section{Vpopmail}

Vpopmail es un software que permite la gesti�n de dominios virtuales. Este
software va a usar MySQL como almac�n de datos. 

\subsection{Instalaci�n}

Excepto la descompresi�n de las fuentes, todo el proceso de instalaci�n de
este componente de software debe de hacerse como root. El programa puede
bajarse desde \emph{http://vpopmail.sf.net}\footnote{La �ltima versi�n estable
es la 5.4.0}

\begin{enumerate}
\item Bajar las fuentes de vpopmail. 
\item Descomprimir las fuentes.
\begin{verbatim}
gunzip -c vpopmail-5.4.0.tar.gz | tar xvf -
\end{verbatim}
\item Crear el usuario y grupo necesario para el funcionamiento del vpopmail:
\begin{verbatim}
groupadd vchkpw
# Para el caso de Redhat
useradd -M -s /sbin/nologin  -g vchkpw vpopmail
# Para el caso de Debian
useradd -M -s /bin/false -g vchkpw vpopmail
\end{verbatim}
\item Configurar, compilar e instalar el vpopmail con soporte para MySQL. Esta
orden instalar� por defecto en /home/vpopmail
\begin{verbatim}
./configure --enable-auth-module=mysql
make
make install
\end{verbatim}
\item Configurar el acceso a la base de datos (login, password y base de
datos) en el fichero \emph{/home/vpopmail/etc/vpopmail.mysql}
\begin{verbatim}
localhost|0|vuser|vclave|vdatabase
\end{verbatim}
Donde \emph{vuser} es el usuario para acceder a la base de datos,
\emph{vclave} es la clave de dicho usuario y \emph{vdatabase} es la base de
datos que se va a usar.
\end{enumerate}

Tras seguir este proceso, el vpopmail quedar� instalado en
\emph{/home/vpopmail}. La estructura de directorios que queda es:

\begin{tabular}{|l|l|}
\hline
/home/vpopmail/bin & Programas ejecutables \\
/home/vpopmail/doc & Documentaci�n \\
/home/vpopmail/domains & Almacen de correo \\
/home/vpopmail/etc & Ficheros de configuracion \\
/home/vpopmail/include & Includes para desarrolo \\
/home/vpopmail/lib & Librerias para desarrollo \\
\hline
\end{tabular}
