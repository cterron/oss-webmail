\section{SquirrelMail}

La aplicaci�n de webmail que va a usarse  est� escrita completa en PHP. Para
consultar el correo que se mostrar� a los usuarios utiliza IMAP, mientras que
para enviar, usa SMTP a trav�s del propio Qmail instalado. Este modo de
funcionamiento tiene la ventaja de que puede separarse f�cilmente la parte de
webmail a otra m�quina.

\subsection{Instalacion}

Debe de bajarse la �ltima versi�n disponible, que puede encontrase en \emph{
http://www.squirrelmail.org/download.php}. En el momento de escribir este
documento la versi�n disponible es 1.2.11.

El punto donde va a instalarse es \emph{/var/www/webmail}, estando ya Apache
configurado para trabajar desde ese directorio.

Los pasos a seguir por la instalaci�n son:
\begin{itemize}
\item Descomprimir las fuentes:
\begin{verbatim}
gunzip -c squirrelmail-1.2.11.tar.gz | tar xvf -
\end{verbatim}
\item Cambiar al usuario root. Ir al directorio donde se ha descomprimido las
fuentes y copiarlas al sitio donde va a ir el webmail:
\begin{verbatim}
cd squirrelmail-1.2.11
cp -r * /var/www/webmail
\end{verbatim}
\item Establecer los permisos adecuados. El root debe ser propietario de todos
los ficheros, exceto del directorio \emph{/var/www/webmail/data} que debe ser
el usuario y grupo sobre el que se ejecuta Apache. Para ello:
\begin{verbatim}
cd /var/www/webmail
chown -R root.root *
chown -R apache.apache data
\end{verbatim} 
\end{itemize}

\subsection{Configuraci�n}

El webmail tiene un peque�o programa a trav�s del cual se configura. Para ello
en el directorio donde se ha instalado, se ejecuta ./configure.
 El menu que sale es el siguiente:
\begin{verbatim}
SquirrelMail Configuration : Read: config_default.php (1.2.0)
---------------------------------------------------------
Main Menu --
1.  Organization Preferences
2.  Server Settings
3.  Folder Defaults
4.  General Options
5.  Themes
6.  Address Books (LDAP)
7.  Message of the Day (MOTD)
8.  Plugins
9.  Database

D.  Set pre-defined settings for specific IMAP servers

C.  Turn color on
S   Save data
Q   Quit
\end{verbatim}

No olvidar una vez realizado todos los cambios, grabarlos

Debe de configurarse:
\begin{itemize}
\item El nombre de la organizaci�n y lenguaje por defecto. Para ello, opci�n
1, y en menu que aparece:
\begin{verbatim}
SquirrelMail Configuration : Read: config_default.php (1.2.0)
---------------------------------------------------------
Organization Preferences
1.  Organization Name      : SquirrelMail
2.  Organization Logo      : ../images/sm_logo.png
3.  Org. Logo Width/Height : (308/111)
4.  Organization Title     : SquirrelMail $version
5.  Signout Page           :
6.  Default Language       : en_US
7.  Top Frame              : _top

R   Return to Main Menu
C.  Turn color on
S   Save data
Q   Quit

Command >>

\end{verbatim}
Camnbiamos \emph{Organization Name} y \emph{Organization Title} a lo que se desee, en
nuestro caso Reinos Olvidados, y el \emph{Default Language} a \emph{es\_ES}.

\item Cambiar la configuraci�n del servidor de correo. Para ello desde el men�
principal, opci�n 2, sale el siguiente menu:
\begin{verbatim}
SquirrelMail Configuration : Read: config_default.php (1.2.0)
---------------------------------------------------------
Server Settings
1.  Domain               : mydomain.com
2.  IMAP Server          : localhost
3.  IMAP Port            : 143
4.  Use Sendmail/SMTP    : SMTP
6.    SMTP Server        : localhost
7.    SMTP Port          : 25
8.    Authenticated SMTP : false
9.    POP Before SMTP    : false
10. Server               : cyrus
11. Invert Time          : false
12. Delimiter            : detect

R   Return to Main Menu
C.  Turn color on
S   Save data
Q   Quit

Command >>
\end{verbatim}

Debe de cambiarse \emph{Domain} al dominio primario que se est� usando. Debe
de cambiarse el tipo de \emph{Server} (opci�n 11) a \emph{courier}.
\end{itemize}
