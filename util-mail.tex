\section{Utilidades de mail}

En esta secci�n se hablar� de como instalar algunas de las utilidades que nos
aportar�n funcionalidad a todo el sistema: gestores de listas de correos y 
autoresponders.

\subsection{ezmlm}

Este es un software del mismo autor del qmail, que permite una f�cil creaci�n
de listas de correo. Permite que cada usuario sea capaz de crear una lista sin
tener que estar molestando a los administradores del sistema. 

\subsubsection{Instalaci�n}

La �ltima versi�n disponible es la 0.53, que puede bajarse de la url
\emph{http://cr.yp.to/software/ezmlm-0.53.tar.gz}. Los ficheros de
configuraci�n por defecto son \emph{conf-qmail}, \emph{conf-bin} y
\emph{conf-man}, que indican donde est� situado qmail, donde ir�n los binarios
del gestor de lista y donde ir�n las p�ginas de manual. En este trabajo se
quedar� todo en su sitio por defecto. 

Para instalar el programa deben seguirse  los siguientes pasos:

\begin{itemize}
\item Descomprimir las fuentes, pasar al directorio resultante y compilar las
fuentes:
\begin{verbatim}
gunzip -c  ezmlm-0.53.tar.gz | tar xvf-
cd ezmlm-0.53
make
make man
\end{verbatim}
\item Como \emph{root}, instalar los binarios:
\begin{verbatim}
make setup
\end{verbatim}
\end{itemize}

\subsection{Autoresponder}

Este peque�o programa sirve para generar respuestas autom�ticas junto con
Qmail. 

\subsubsection{Instalaci�n}

El programa puede bajarse de la url
\emph{http://www.inter7.com/devel/autorespond-2.0.2.tar.gz}. Necesita saber la
localizaci�n del directorio de Qmail. Sin embargo, como se est� usando la
instalaci�n por defecto, no es necesario modificar nada. El programa se
instalar� por defecto en \emph{/usr/local/bin}.  Para instalarlo deben
de seguirse los siguientes pasos:
\begin{itemize}
\item Descomprimir las fuentes, cambiar al directorio resultante y compilar
las fuentes:
\begin{verbatim}
gunzip -c autorespond-2.0.2.tar.gz | tar xvf - 
make
\end{verbatim}
\item Como \emph{root}, instalar el programa
\begin{verbatim}
make install
\end{verbatim}
\end{itemize}
