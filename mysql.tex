\section{Base de datos}
La  base de datos usada va a ser MySQL. En ella se almacenar� la informaci�n
necesaria de los dominios y usuarios.

\subsection{Instalaci�n}

Hay que instalar tanto el motor de base de datos como el soporte necesario
para desarrollar. Esto debe hacerse antes de compilar el vpopmail.

Lo primero que debe de hacerse una vez instalado es darle una password de
administrador al servidor. Una vez que el servidor de base de datos est�
instalado, debemos de asegurarnos que el servidor est� arrancado\footnote{En
Redhat se arranca con /etc/init.d/mysqld start} y darle una password al
usuario de root:
\begin{verbatim}
mysqladmin password passwordnueva
\end{verbatim}
Donde \emph{passwordnueva} es la clave que se le da al administrador de la
misma.

Ahora debe de crearse la base de datos que va a usarse para el vpopmail, crear
el usuario para que el vpopmail acceda a la misma y darle los permisos
adecuados:
\begin{enumerate}
\item Conectarse a la base de datos con el usuario de root. El sistema
preguntar� por la clave dicho usuario y una vez que se haya autentificado
correctamente, nos presentar� el prompt de mysql:
\begin{verbatim}
mysql -u root -p
\end{verbatim}
\item Se crea la base de datos y el usuario, con las siguientes ordenes en el
prompt de mysql. Si el usuario que accede a la base de datos es \emph{vpopmail}
 y la clave del mismo es \emph{myclave}
\begin{verbatim}
create database vpopmail;
grant all on vpopmail.* to vpopmail identified by 'myclave';
flush privileges;
\end{verbatim}
\end{enumerate}
