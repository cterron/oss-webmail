\section{Apache}

Apache es el servidor web que se va usar en el proyecto para dar soporte a un
webmail y a una peque�a utilidad de administraci�n de los diferentes dominios.
Hay que asegurarse que se est� instalando la �ltima versi�n disponible para
la distribuci�n, puesto que hay ciertos problemas de seguridad serios que
est�n resueltos en ellos.

En esta secci�n tambi�n se describe como instalar el cgi \emph{qmailadmin} que
va a usarse para configurar los dominios virtuales una vez creados.

Debe de decidirse tambi�n el puerto donde va escuchar el servicio de
administraci�n. Es aconsejable, si se dispone de dos interfaces de red, una
abierta al exterior y otra de administraci�n, la parte que se encarga de la
administraci�n debe de escuchar en la interna.

En esta secci�n no entrar� en profundidad en las diferentes opciones de
configuraci�n de Apache, puesto que la complejidad del mismo puede abarcar m�s
de un libro. Toda la documentaci�n del servidor web puede encontrarse en la 
\emph{http://httpd.apache.org/docs/}.

Se van a crear dos sitios web, uno de ellos se utilizar� para  el webmail,
mientras que el otro se utilizar� para la administraci�n del sitio. En este
ejemplo en la web principal va a estar el webmail y la zona de administraci�n
se va configurar en otra direcci�n IP y en un puerto distinto del 80. Los
datos pueden consultarse en la maqueta. 

El fichero de configuraci�n de Apache se encuentra en la Redhat 7.3 en
\emph{/etc/httpd/conf/httpd.conf}. Este archivo viene por defecto viene
configurado con un mont�n de opciones para la Redhat.  Se comentar�n las
opciones que deben a�adirse o modificarse.

Los pasos a seguir para configurar Apache son los siguientes.

\begin{itemize}
\item Crear los directorios donde va a colocarse el webmail, la utilidad de
administraci�n y los ficheros de logs. Los pasos a dar son:
\begin{verbatim}
mkdir -p /var/log/httpd
mkdir -p /var/log/httpd/webmail
mkdir -p /var/log/httpd/qmailadmin
mkdir -p /var/www/webmail
mkdir -p /var/www/qmailadmin
mkdir -p /var/www/qmailadmin/cgi-bin
\end{verbatim}
\item Modificar la parte global de Apache con las opciones siguientes:
\begin{itemize}
	\item A�adir las siguientes lineas para que escuche en los puertos
necesarios:
\begin{verbatim}
Listen 80
Listen 192.168.0.1:2000
\end{verbatim}	
	\item Cambiar la ra�z del documento al webmail. Para ello, localizar
	la l�nea:
\begin{verbatim}
DocumentRoot "/var/www/html"
 cambiarla a
DocumentRoot "/var/www/webmail"
 Localizar las linea
<Directory "/var/www/html"> y cambiarla por
<Directory "/var/www/webmail">. 
\end{verbatim}
Con esto se consigue que el site principal de Apache sea donde va a estar
situado el webmail.
	\item A�adir un servidor virtual nuevo en el puerto 2000 y escuchando en
la IP interna. Para ello, al final del fichero a�adir lo siguiente:
\begin{quote}
\bfseries{
\begin{verbatim}
<VirtualHost 192.168.0.1:2000>
		SSLProtocol all
        DocumentRoot "/var/www/qmailadmin"
        ErrorLog "/var/log/httpd/qmailadmin/error_log"
        TransferLog "/var/log/httpd/qmailadmin/access_log"
        ScriptAlias /cgi-bin/ /var/www/qmailadmin/cgi-bin/
        <Directory "/var/www/qmailadmin">
                AllowOverride None
                Options None
                Order allow,deny
                Allow from all
        </Directory>
        ServerName 192.168.0.1
</VirtualHost>
\end{verbatim}
}
\end{quote}
\end{itemize}
\end{itemize}

\subsection{Qmailadmin}

Este CGI puede obtenerse en \emph{http://www.inter7.com/qmailadmin.html}. Esta
utilidad permite gestionar los diferentes dominios que se han creado. La url
para la versi�n disponible en el momento de escribir este documento es 
\emph{http://www.inter7.com/qmailadmin/qmailadmin-1.0.6.tar.gz}

Para
instalarlo deben de seguir los siguientes pasos:

\begin{enumerate}
\item Descomprimir las fuentes. Cambiar al directorio donde se han
descomprimido:
\begin{verbatim}
gunzip -c  qmailadmin-1.0.6.tar.gz | tar xvf -
cd qmailadmin-1.0.6
\end{verbatim}
\item Configurar y compilar el programa, teniendo en cuenta las opciones que se han
creado para Apache en el sitio de administraci�n, es decir que el ra�z del
servicio web virtual est� en /var/www/qmailadmin y que los cgi para dicho
servicio est�n en /var/www/qmailadmin/cgi-bin:
\begin{verbatim}
./configure --enable-htmldir=/var/www/qmailadmin  \ 
         --enable-cgibindir=/var/www/qmailadmin/cgi-bin
make
\end{verbatim} 
\item Como root, instalar el programa:
\begin{verbatim}
make install
\end{verbatim}
\end{enumerate}

El programa de administraci�n podr� accederse a trav�s de
http://192.168.0.1/cgi-bin/qmailadmin. Para entrar basta con introducir el
dominio y la clave que se la asignado al postmaster.

