\section{Courier IMAP}

Este programa va a ser el encargado del protocolo IMAP. Tiene varios m�dulos
de autentificaci�n. Uno de estos m�dulos permite usar las librerias de
vpopmail para obtener la informaci�n necesaria. Adem�s, tiene un m�dulo que
permite cachear la informaci�n de autentificaci�n ahorrando consultas
innecesarias que acaban en la base de datos.

\subsection{Instalaci�n}

Debe de bajarse el courier-imap de la url
\emph{http://prdownloads.sourceforge.net/courier/courier-imap-1.7.0.tar.bz2}.
La configuraci�n de este software no puede hacerse como root. Para ello
debe usarse una cuenta normal. Puede usarse la cuenta \emph{mailadmin} 
para ello. La versi�n a la hora de escribir este documento es la 2.2.1.

En principio este software se va a instalar por defecto en
\emph{/usr/lib/courier-imap}.
Los pasos a seguir son lo siguientes:
\begin{itemize}
\item Con una cuenta \emph{distinta a root}, descomprimir el fichero,
configurar y compilar el programa:
\begin{verbatim}
bzcat courier-imap-2.2.1.tar.bz2 | tar xvf -
cd courier-imap-2.2.1
./configure --with-authpam \
	--sysconfdir=/etc/courier \
	--with-authvchkpwd \
	--without-authldap \
	--without-authmysql
make 
\end{quote}
En el caso de estar instal�ndose sobre Fedora, a�adir la opci�n:
\begin{verbatim}--with-redhat\end{verbatim}
\item Instalar como root, los binarios (desde el directorio donde se ha
compilado el programa).
\begin{verbatim}
make install
\end{verbatim}

\item Crear el fichero que arrancar� el authdaemond, el programa que se
encargar� de la autentificaci�n. Para ello crear el fichero
\emph{/etc/init.d/authdaemond}, con el siguiente contenido:
\begin{verbatim}
#!/bin/sh
#
# chkconfig: 345 95 95 
# description: authdaemond
# processname: authdaemond
prefix=/usr/lib/courier-imap

if [ ! -f /etc/courier/authdaemonrc ]; then
	exit 1
fi
if [ ! -x $prefix/libexec/authlib/authdaemond.plain ]; then
	exit 1
fi

. /etc/courier/authdaemonrc
case "$1" in
	start)
		$prefix/libexec/authlib/authdaemond.plain start
		;;
	stop)
		$prefix/libexec/authlib/authdaemond.plain stop
		;;
	*)
		echo "Usage: authdaemond start|stop"
		exit 1
esac
\end{verbatim}
Darle permiso de ejecuci�n con:
\begin{verbatim}
chmod 755 /etc/init.d/autjdaemond 
\end{verbatim}
En el caso de la Fedora puede a�adirse a los scripts de inicio con la orden:
\begin{verbatim}
/sbin/chkconfig --add authdaemond
\end{verbatim}
\item Crear el directorio donde van a almacencarse los logs de imap. Este
directorio ser� \emph{/var/log/qmail/imapd}. Como root ejecutar las siguiente
�rden:
\begin{verbatim}
mkdir -p /var/log/qmail/imapd
\end{verbatim}
\item Crear el usuario para los logs, que llamaremos \emph{imaplog}. Cambiar
el propietario del directorio del paso anterior a este usuario.
Como root ejecutar:
\begin{verbatim}
useradd -M -d /var/log/qmail/imapd imaplog
useradd -M -d /var/log/qmail/pop3d pop3log
useradd -M imapd
useradd -M pop3d
chown imaplog /var/log/qmail/imapd
chown pop3log /var/log/qmail/pop3d
\end{verbatim}

\item Crear los directorios para el servicio y para los logs.
Establecer los permisos adecuados:
\begin{verbatim}
mkdir -p /usr/lib/courier-imap/supervise/imap
mkdir -p /usr/lib/courier-imap/supervise/imap/log
chmod 1755 /usr/lib/courier-imap/supervise/imap/log
mkdir -p /usr/lib/courier-imap/supervise/pop3
mkdir -p /usr/lib/courier-imap/supervise/pop3/log
chmod 1755 /usr/lib/courier-imap/supervise/pop3/log
\end{verbatim}
\item Crear el script \emph{/usr/lib/courier-imap/supervise/imap/run}  con el
siguiente contenido:
\begin{verbatim}
#!/bin/sh
USERIMAP=`id -u imapd`
GROUPIMAP=`id -g imapd`
exec \
/usr/local/bin/tcpserver -u "$USERIMAP" -g "$GROUPIMAP" \ 
-h -R 0 imap \
/usr/lib/courier-imap/sbin/imaplogin \
/usr/lib/courier-imap/libexec/authlib/authdaemon \
/usr/lib/courier-imap/bin/imapd Maildir 2>&1
\end{verbatim}

\item Crear el script \emph{/usr/lib/courier-imap/supervise/imap/log/run} con el
siguiente contenido:
\begin{verbatim}
#!/bin/sh
exec \
/usr/local/bin/setuidgid imaplog \
/usr/local/bin/multilog t \
/var/log/qmail/imapd
\end{verbatim}

\item Crear el fichero \emph{/usr/lib/courier-imap/supervise/pop3/run}
\begin{verbatim}
#!/bin/sh
USERIMAP=`id -u pop3d`
GROUPIMAP=`id -g pop3d`
exec \
/usr/local/bin/tcpserver -u "$USERIMAP" -g "$GROUPIMAP" \ 
-h -R 0 pop3 \
/usr/lib/courier-imap/sbin/pop3login \
/usr/lib/courier-imap/libexec/authlib/authdaemon \
/usr/lib/courier-imap/bin/pop3d  Maildir 2>&1
\end{verbatim}

\item Crear el fichero \emph{/usr/lib/courier-imap/supervise/pop3/log/run}:
\begin{verbatim}
#!/bin/sh
exec \
/usr/local/bin/setuidgid pop3log \
/usr/local/bin/multilog t \
/var/log/qmail/pop3d
\end{verbatim}


\item Dar permisos de ejecuci�n a los scripts anteriores y a�adir al sistema
de arranque del supervise:
\begin{verbatim}
chmod 755 /usr/lib/courier-imap/supervise/imap/run
chmod 755 /usr/lib/courier-imap/supervise/imap/log/run
ln -s /usr/lib/courier-imap/supervise/imap  /service/imap
chmod 755 /usr/lib/courier-imap/supervise/pop3/run
chmod 755 /usr/lib/courier-imap/supervise/pop3/log/run
ln -s /usr/lib/courier-imap/supervise/pop3 /service/pop3
\end{verbatim}

\item Quitar los permisos en los directorios que se han dado para compilar.
\begin{quote}
\bfseries{
chmod 700 /home/vpopmail/lib
\newline
chmod 600 /home/vpopmail/lib/libvpopmail.a
}
\end{quote}
\end{itemize}

En caso de problemas debe de mirarse si el proceso ha arrancado correctamente
y los archivos de registro creado por el mismo.
