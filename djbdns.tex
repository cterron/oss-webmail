\section {Cach� de DNS}

Se va a usar un cach� de DNS para acelerar las consultas a DNS. El software
elegido es el djbdns. Debe de ternerse instalada las deamontools para poder 
ejecutar correctamente este software.

\subsection{Instalaci�n}

La p�gina principal del Djbdns es \emph{http://cr.yp.to/djbdns.html}. 
El software puede bajarse de la siguiente direcci�n:\newline
\emph{http://cr.yp.to/djbdns/djbdns-1.05.tar.gz}. Para instalar
este software deben de seguirse los siguientes pasos:
\begin{enumerate}
\item Descomprimir el fichero: 
	\begin{quote}
\begin{verbatim}
gunzip -c djbdns-1.05.tar.gz | tar xvf -
\end{verbatim}
	\end{quote}
\item Ejecutar la siguiente orden para compilar el programa:
	\begin{quote}
	\begin{verbatim}
make
\end{verbatim}	
	\end{quote}
\item Como superusuario ejecutar la siguiente orden para instalar el programa
(por defecto se instala en \emph{/usr/local}.
	\begin{quote}
\begin{verbatim}	
make setup check
\end{verbatim}
\end{quote}
\item Crear el directorio \emph{/etc/dnscache} 
	\begin{quote}
\begin{verbatim}
mkdir -p /etc/dnscache
\end{verbatim}
	\end{quote}
\item A�adir los usuarios \emph{dnscache} y \emph{dnslog}. Para ello ejecutar
las siguientes �rdenes:
\begin{quote}
\begin{verbatim}
useradd dnscache
useradd dnslog
\end{verbatim}
\end{quote}
\item Ejecutar la siguiente orden para generar los ficheros de configuraci�n
del cach� de DNS:
\begin{quote}
\begin{verbatim}
dnscache-conf dnscache dnslog /etc/dnscache
\end{verbatim}
\end{quote}
\item Crear el directorio \emph{/var/log/dnslog} y cambiarle el propietario al
usuario \emph{dnslog}. Para ello deben de ejecutarse las siguientes ordenes
(como root):
	\begin{quote}
\begin{verbatim}
mkdir -p /var/log/dnslog
chown dnslog /var/log/dnslog
\end{verbatim}
	\end{quote}
\item Cambiar el sitio donde almacena los logs el cach�. Para ello debe de
cambiarse el script \emph{/etc/dnscache/log/run} al siguiente:
\begin{verbatim}
#!/bin/sh
MAXLOGSIZE=1048576
NUMLOG=7
exec setuidgid dnslog ssize $MAXLOGSIZE nnum $NUMLOG multilog t /var/log/dnscache
\end{verbatim}
\item Crear los enlaces simb�licos en el directorio \emph{/service}. 
\begin{quote}
\begin{verbatim}
	ln -s /etc/dnscache /service/dnscache
\end{verbatim}
\end{quote}
\item Cambiar el fichero \emph{/etc/resolv.conf} de tal manera que solo quede
la linea:
\begin{quote}
\begin{verbatim}
nameserver 127.0.0.1
\end{verbatim}
\end{quote}
\item Comprobar el correcto funcinamiento del cach�. Para ello resolver
algunos nombres con dnsip:
\begin{quote}
\begin{verbatim}
dnsip www.slashdot.org
dnsip www.barrapunto.com	
\end{verbatim}
\end{quote}
En caso de que exista alg�n problema pueden consultarse los logs que est�
generando el cach� en \emph{/var/log/dnscache/current}\footnote{En el apendice
XXX se hace referencia a este formato de logs}
\end{enumerate}

Es importante para el correcto funcionamiento del cach� que en el PATH de
b�squeda del sistema est� incluido el directorio \emph{/usr/local}. En caso de
que no sea as� siempre se puede editar los ficheros \emph{/etc/dnscache/run} y
\emph{/etc/dnscache/log/run} para usar caminos absolutos hacia los
ejecutables.

Debe de mirarse la secci�n del texto que explican el funcionamiento de las
daemontools para usarlas correctamente y comprender exactamente el proceso de
arranque de los servicios que las usan.
